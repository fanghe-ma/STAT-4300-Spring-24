\chapter{Class 2}

\section{Counting}

\subsection{Multiplication rule}
\begin{framed}
   Consider compound experiment, consisting of $2$ sub-experiments. Experiment 1 has $n$ outcomes, experiment 2 as $n$ outcomes. Then the compound experiment has $mn$ outcomes
\end{framed}

\subsection{Sampling with / without replacement}

Sampling $k$ from $n$ with replacement
\[
  n^k
\] 

Sampling $k$ from $n$ without replacement
\[
  \frac{n!}{(n-k)!}
\] 

\subsection{Birthday Paradox}

The probability of having a birthday match is
\begin{align*}
   P(\text{birthday match}) &= \frac{\text{Number of outcomes that match}}{ \text{Number of total outcomes}} \\
                            &= 1 - \frac{ \text{ Number of outcomes without match}}{ \text{Number of total outcomes}} \\
                            &= 1 - \frac{365!}{(365 - k)! 365^k}
\end{align*}


\subsection{Binomial coefficients}
The number of ways to choose $k$ out of $n$ can be counted by
\begin{enumerate}
   \item multiplication rule 
      \[
        n \times n -1 \times \hdots \times (n - k+1) = \frac{n!}{(n-k)!}
      \] 
   \item adjust for overcounting (account for $k!$ reorderings)
      \[
         \frac{1}{k!} \frac{n!}{(n-k)!} = \frac{n!}{k!(n-k)!}
      \] 
\end{enumerate}
More generally, 
\[
  \begin{pmatrix} n \\ k \end{pmatrix}  = \frac{n (n-1) \hdots (n - k + 1)}{k!} = \frac{n!}{k! (n-k)!}
\] 

\subsection{Example with Poker}

Find the probability of a royal flush \\
\[
  P(\text{ Royal flush }) = \frac{4}{ \begin{pmatrix} 52 \\5 \end{pmatrix} } 
\] 

Find the probability of a flush
\[
  P(\text{ Flush }) =  \frac{4 \begin{pmatrix} 13 \\5 \end{pmatrix} }{ \begin{pmatrix} 52 \\5 \end{pmatrix} } \approx 0.002
\] 

Find the probability of a full house
\[
  P(\text{ Full house }) =  \begin{pmatrix} 13 \\2 \end{pmatrix}  \begin{pmatrix} 2 \\ 1 \end{pmatrix}  \begin{pmatrix} 4 \\3 \end{pmatrix}  \begin{pmatrix} 4 \\2 \end{pmatrix} \div \begin{pmatrix} 52 \\5  \end{pmatrix}  \approx 0.001
\] 
\section{A refined definition of probability}

Previous definition: 
\[
   \mathbb{P}_{ \text{Naive}} \text{ is a function where} \left \{ \begin{array}{ll} 
         \text{inputs} \\
         \text{outputs}
   \end{array} \right.
\] 

\begin{framed}
   \textbf{Definition}\\

   A \textbf{probability} space consists of
   \begin{itemize}
      \item Sample space $S$
      \item Probability function $P$ 
         \begin{itemize}
            \item Input: any event $A \in S$
            \item Output: real number $P(A) \in [0, 1]$
         \end{itemize}
   \end{itemize}

   Axiomatic definitions of probability function
   \begin{itemize}
      \item $P(\emptyset) = 0$
      \item $P(S) = 1$
      \item For $A_1, A_2, \hdots$ disjoint, then
         \[
            P( \bigcup_{k=1}^{\infty}A_k) = \sum_{k = 1}^{\infty}P(A_k)
         \] 
   \end{itemize}
\end{framed}
